\begin{frame}{Configuration and vibration}
	\begin{columns}
	\column{0.6\textwidth}
	\def\svgwidth{\columnwidth}
	\centering\small{\input{specific_heat.pdf_tex}}
	
	\column{0.4\textwidth}
	A solid has only one configuration: non-ergodic
	\[ S = S_c + S_{vib} \]
	\end{columns}
\end{frame}

\begin{frame}{Dynamical functions}
	\begin{itemize}
		\item Self (Incoherent) Intermediate Function
		\[ F_s(q,t) \equiv \frac{1}{N} \left\langle\sum_{i=1}^N e^{-\imath\vec{q}\cdot\bigl(\vec{r_i}(t)-\vec{r_i}(0)\bigr)}\right\rangle \]
		\item Mean square displacement
		\[ \Delta r(t)^2 = \left\langle\| \vec{r_i}(t)-\vec{r_i}(0)\|^2\right\rangle \]
	\end{itemize}
\end{frame}

\section{Structure}

\begin{frame}{Local structures}
	\begin{center}\begin{tabular}{ccc}
	\textsc{fcc} & \textsc{hcp} & Icosahedron \\ 
	\includegraphics[width=0.27\textwidth]{fcc_13} & \includegraphics[width=0.27\textwidth]{hcp_13} & \includegraphics[width=0.27\textwidth]{ico_13}
	\end{tabular}
	\only<all:1>{\begin{tabular}{ccc}
	\includegraphics[width=0.27\textwidth]{bcc_9} & \includegraphics[width=0.27\textwidth]{bcc_15} & \includegraphics[width=0.27\textwidth]{dodec_13} \\ 
	\textsc{bcc} 9 & \textsc{bcc} 15 & Dodecahedron \\ 
	\end{tabular}}%
	\end{center}
	\only<all:2>{%
	\begin{itemize}
		\item The best ways to pack particles
		\item In hard spheres
		\begin{itemize}
			\item no potential energy
			\item ordering maximizes local vibrations
			\item vibrational entropy $\Leftrightarrow$ configurational entropy
			\item packing drives ordering
		\end{itemize}
	\end{itemize}
	
	\bigskip}%
\end{frame}

\subsection{Bond network}

\begin{frame}{Bond network - Neighbours}
	\begin{columns}
	\column{0.4\textwidth}
	\includegraphics[width=\columnwidth]{voro2d}\\
	\centering Voronoi decomposition
	\column{0.6\textwidth}
	\centering Radial distribution function\\
	{\def\svgwidth{\columnwidth}\input{typicalRdf.pdf_tex}}
	\begin{itemize}
		\item A bond between neighbours
		\item Voronoi is not good (anisotropy)
		\item First shell
	\end{itemize}
	\end{columns}
\end{frame}

\begin{frame}{Topology of the bond network}
	\begin{columns}
	\column{0.25\textwidth}
	\centering \textsc{fcc}\\
	\includegraphics[width=\columnwidth]{fcc_13}
	
	\bigskip
	\includegraphics[width=\columnwidth]{ico_13_1551}\\
	Icosahedron
	\column{0.7\textwidth}
	\begin{itemize}
		\item Number of neighbours
		\item Voronoi signature \citet{tanemura1977geometrical}
		\item Common neighbours \citet{Honeycutt1987}
		\item Topological cluster classification \citet{Williams2007}
	\end{itemize}
	Difficult to correlate discrete categories in time or space.
	\end{columns}
\end{frame}

\subsection{Spherical harmonics}

\begin{frame}{Spherical harmonics}
	\begin{columns}
	\column{0.5\textwidth}
	\begin{itemize}
		\item Analogue to Fourier decomposition
		\item On a sphere
	\end{itemize}
	\column{0.5\textwidth}
	\[ h(\theta,\phi) = \sum_{\ell=0}^{\infty} \sum_{m=-\ell}^{\ell} q_{\ell m} Y_{\ell m}(\theta,\phi) \]
	\end{columns}
	\begin{description}
		\item[$\ell$] Order of symmetry
		\item[$m$] Orientation
		\item[$(r,\theta,\phi)$] Spherical coordinates
	\end{description}
	\begin{center}Altitude on earth: Decomposition\end{center}
	\begin{tabular}{ccc}
	\includegraphics[width=0.29\textwidth]{earth_l1} & \includegraphics[width=0.29\textwidth]{earth_l2} & \includegraphics[width=0.29\textwidth]{earth_l3} \\ 
	$\ell=1$ & $\ell=2$ & $\ell=3$ \\ 
	\end{tabular} 
\end{frame}

\begin{frame}{Approximation by spherical harmonics}
	\begin{columns}[T]
	\centering
	\column{0.4\textwidth}
	\includegraphics[width=\textwidth]{earth_to6}\\
	\centering{$\sum_{\ell=0}^6$}
	
	\includegraphics[width=\textwidth]{earth_to36}\\
	\centering{$\sum_{\ell=0}^{36}$}
	\column{0.4\textwidth}
	\includegraphics[width=\textwidth]{earth_to16}\\
	\centering{$\sum_{\ell=0}^{16}$}
	
	\includegraphics[width=\textwidth]{earth_grid}\\
	\footnotesize{Full grid of earth topography}
	\end{columns}
\end{frame}

\subsection{Bond orientational order}

\againframe{local_sym_sh}

\begin{frame}{Bond orientational order}
	\begin{columns}
	\column{0.5\textwidth}
	\begin{itemize}
		\item A set of spherical harmonics for each
		\begin{itemize}
			\item Bond $\vec{r}_{ij}$
			\item Neighbourhood
		\end{itemize}
		\item Rotational invariants
		\begin{itemize}
			\item Strength of the $\ell$-fold symmetry
			\item Characterise the symmetry group
		\end{itemize}
	\end{itemize}
	\column{0.5\textwidth}
	\[ q_{\ell m}(i) = \frac{1}{N_i}\sum_{j=0}^{N_i} Y_{\ell m}(\theta(\vec r_{ij}),\phi(\vec r_{ij})) \]
	\[ q_\ell = \sqrt{\frac{4\pi}{2l+1} \sum_{m=-\ell}^{\ell} |q_{\ell m}|^2 } \]
	\end{columns}
	\[w_\ell = \sum_{m_1+m_2+m_3=0} 
			\left( \begin{array}{ccc}
				\ell & \ell & \ell \\
				m_1 & m_2 & m_3 
			\end{array} \right)
			q_{\ell m_1} q_{\ell m_2} q_{\ell m_3}\]
	\[\hat{w}_\ell = w_\ell{\left( \sum_{m=-\ell}^{\ell} |q_{\ell m}|^2 \right)}^{-\frac{3}{2}}\]
	
	\footnotesize{\citet{steinhardt1983boo}}
\end{frame}

\begin{frame}{Invariants distributions}
	\begin{columns}
	\column{0.6\textwidth}
	\includegraphics[width=\columnwidth]{gasser_invariants.jpg}
	\column{0.4\textwidth}
	\begin{itemize}
		\item Distributions are
		\begin{itemize}
			\item Noisy
			\item Broad
			\item Overlapping
		\end{itemize}
		\item Can characterize a sample
		\item Cannot characterize a single particle
	\end{itemize}
	\end{columns}
	\footnotesize{\citet{Gasser2001}}
\end{frame}

\begin{frame}{Coarse-grained BOO}
	\begin{columns}
	\column{0.6\textwidth}
	\includegraphics[width=\columnwidth]{invariants_maps_raster}
	\column{0.4\textwidth}
	\[ Q_{\ell m}(i) = \frac{1}{\tilde{N} (i)} \sum_{k=0}^{\tilde{N}(i)} q_{\ell m}(k) \]
	\begin{itemize}
		\item Take into account the second shell
		\item Structures with some periodicity are much better defined
		\item Non periodic structures goes to zeros
	\end{itemize}
	\end{columns}
	\footnotesize{\citet{lechner2008}}
\end{frame}

\begin{frame}{Crystal-like bond order}
	\begin{textblock*}{0.6\textwidth}(10mm,88mm)
		\simplephasediagram{\node at (0.576,0) [xp marker, fill=green!50!black] {};}
	\end{textblock*}
	\begin{columns}
	\column{0.7\textwidth}
	\begin{tikzpicture}
		\pgfplotsset{
			extra tick style={grid=major},%
			every axis/.append style={%
				height=0.55\textwidth,
				ymin=0,ymax=0.6,%
				extra y ticks={\Qstar}, extra y tick labels={},%
				enlargelimits=false,axis on top,
				colormap={bw}{gray(0cm)=(1); gray(1cm)=(1); gray(10cm)=(0)},%
				colorbar sampled,%
				},%
		}
		\begin{groupplot}[
			group style={
				group size=2 by 1,%
				yticklabels at=edge left,%
				horizontal sep=0pt,%
				},%
			anchor=below south west,%
			width=0.55\columnwidth,%
			xtick scale label code/.code={},%
			colorbar horizontal, colorbar style={%
				samples=6, xtick={ 0.20,0.4,0.6, 0.8},% 
				extra y ticks={},%
				/pgfplots/colorbar shift/.style={yshift=0.3cm},
				at={(parent axis.north)}, anchor=below south, width=0.9*\pgfkeysvalueof{/pgfplots/parent axis width},
				xticklabel pos=upper,%
				label style={font=\footnotesize},
				},%
			]
		\nextgroupplot[%
			ylabel={$Q_6$}, xlabel=$Q_4$,%
			xmin=0,xmax=0.22,%
			colorbar style={%
				xlabel={per units of $Q_4\cdot Q_6$},% 
				xticklabels={$10^{1}$, $10^{2}$, $10^{3}$, $10^{4}$},%
				},%
			xticklabel={$\pgfmathprintnumber[fixed,precision=2]{\tick}$}
			]
		\addplot graphics
		[xmin=0,xmax=0.2,ymin=0,ymax=0.6]
		{Q4Q6go1};
		\node [below] at (axis cs:0.1909, 0.5745) {\textsc{fcc}};
		\node [below] at (axis cs:0.0972222, 0.484762) {\textsc{hcp}};
		\node [below] at (axis cs:0.0363696, 0.510688) {\textsc{bcc}};
		\draw[->, white,thick] (axis cs:0.05, 0.15) to [out=60, in=220] (axis cs:0.125, 0.4);
		
		\nextgroupplot[%
			xlabel=$10^3 \cdot w_4$, %
			xmin=-0.002,xmax=0.002,%
			xtickmin=-0.0015,%
			extra x ticks=0, extra x tick labels={},
			colorbar style={%
				xlabel={per units of $w_4\cdot Q_6$},% 
				xticklabels={$10^{3}$, $10^{4}$, $10^{5}$, $10^{6}$},%
				},%
			]
		\addplot graphics
		[xmin=-0.002,xmax=0.002,ymin=0,ymax=0.6]
		{w4Q6go1};
		\node [below] at (axis cs:-0.00067221, 0.5745) {\textsc{fcc}};
		\node [below] at (axis cs:7.47E-05, 0.484762) {\textsc{hcp}};
		\node [above] at (axis cs:0.0015, \Qstar) {\footnotesize{$Q_6^*$}};
		\draw[->, white,thick] (axis cs:0, 0.15) to [out=120, in=280] (axis cs:-0.0005, 0.4);
		\draw[->, white,thick] (axis cs:0, 0.15) to (axis cs:7E-05, 0.35);
		
		\end{groupplot}
	\end{tikzpicture}
	\column{0.3\textwidth}
	$w_\ell$ indicates how the $\ell$-fold symmetry is "rotating"

	\bigskip	
	
	\begin{itemize}
		\item No \textsc{bcc}
		\item Mainly \textsc{fcc}
		\item Some \textsc{hcp}
	\end{itemize}
	
	\bigskip
	
	We keep $Q_6$ as crystal axis
	\end{columns}
\end{frame}

\subsection{Lifetime}

\begin{frame}{Time correlation}
	\begin{itemize}
	\item Time auto-correlation of the bond order
	\[
	g_\ell(t) \equiv \left\langle \frac{
		\sum_{m=-\ell}^{\ell} q_{\ell m}(i, t_0) q_{\ell m}^{*}(i, t_0+t)
	}{
		\sqrt{\sum_{m=-\ell}^{\ell} \left\|q_{\ell m}(i,t_0)\right\|^2}
	}\right\rangle_{i, t_0}
	\]
	\item Can be done with the $Q_{\ell m}$ to discard aperiodic structures $\Longrightarrow G_\ell(t)$
	\item The ratio $G_\ell(t)/g_\ell(t)<1$ gives the proportion of $t$-lived structures that are periodic
	\end{itemize}
\end{frame}

\begin{frame}{Lifetimes}
	\begin{textblock*}{0.6\textwidth}(10mm,88mm)
		\simplephasediagram{}
	\end{textblock*}
	%\tikz[baseline, remember picture]\node[anchor=base] (text)%
	%		{Near $\phi_g$ and $\tau_\alpha$, MRCO live longer than icosahedra};
	\tikzstyle{background grid}=[draw, black!50,step=0.1\textwidth]
	\begin{center}
    \begin{tikzpicture}%, show background grid]
		\node [inner sep=0pt,above right] 
			{\resizebox{0.7\textwidth}{!}{\input{Qlm_qlm_correl_ratio}}};
		%\node [rectangle, red, minimum width=0.08\textwidth, minimum height=0.05\textwidth, draw] at (0.3\textwidth, 0.4\textwidth) (q4) {};
		\node [rectangle, red, minimum width=0.08\textwidth, minimum height=0.05\textwidth, draw] at (0.6\textwidth, 0.4\textwidth) (q6) {};
		\node at (0.35\textwidth, 0.525\textwidth) (text)%
			{Near $\phi_g$ and $\tau_\alpha$, MRCO live longer than icosahedra};
		%\path[->] (text) edge (q4);
		\path[->] (text.east) edge [out=-45, in=0] (q6.east);
	\end{tikzpicture}
	\begin{description}
		\item[$\ell=8\nearrow$] No long-lived aperiodic $8$-fold
		\item[$\ell=10\searrow$] No long lived periodic $10$-fold
	\end{description}
	\end{center}
\end{frame}

\subsection{Spatial correlation}

\begin{frame}{Spatial correlation}
	\begin{itemize}
		\item Correlation of the bond order between particles $i$ and $j$
		\[ s_\ell(i,j) = \frac{
			\sum_{m=-\ell}^{\ell} q_{\ell m}(i) q_{\ell m}^{*}(j)
		}{
			\sqrt{\sum_{m=-\ell}^{\ell} |q_{\ell m}(i)|^2} \sqrt{\sum_{m=-\ell}^{\ell} |q_{\ell m}(j)|^2}
		}\]
		\item Spatial correlation
		\[ g_\ell(r) \equiv \frac{1}{N}\sum_i^N \frac{
			\sum_{j \neq i}{ s_\ell(i,j) \delta\left(\left\|\vec{r}_i-\vec{r}_j \right\| - r \right)}
		}{
		\sum_{j \neq i}{\delta\left(\left\|\vec{r}_i-\vec{r}_j \right\| - r \right)}
		} \]
		\item Can be done with the $Q_{\ell m}$ to discard aperiodic structures $\Longrightarrow G_\ell(r)$
	\end{itemize}
\end{frame}

\begin{frame}{Two correlation lengths}
	\begin{textblock*}{0.6\textwidth}(10mm,88mm)
		\simplephasediagram{}
	\end{textblock*}
	\begin{tikzpicture}
		\begin{groupplot}[%
			group style={
				group size=2 by 1,%
				horizontal sep=0, vertical sep=0.5em,%
				xlabels at=edge bottom, xticklabels at=edge bottom,%
				},%
			width=0.45\textwidth,%
			xlabel near ticks, xlabel shift=-0.3em,%
			legend pos=south west, reverse legend, legend style={font=\tiny}, %
			]
			\nextgroupplot[%
				xmin=1.8, xmax=6,%
				ytick pos=left, ylabel near ticks,%
				ylabel=$G_6(r)$, ymin=1e-5, ymode=log,%
				]
			\addplot+[mark=none, forget plot, domain=0:6] {0.06983353/x*exp(-x/0.90777071)};
			\addplot+[only marks] table[
				x index=0, y index=1,%
				] {LS4446.cg6};
			\addplot+[mark=none, forget plot, domain=0:6] {0.05273075/x*exp(-x/1.05925359)};
			\addplot+[only marks] table[
				x index=0, y index=1,%
				] {LS5079.cg6};
			\addplot+[mark=none, forget plot, domain=0:6] {0.04594942/x*exp(-x/1.51521193)};
			\addplot+[only marks] table[
				x index=0, y index=1,%
				] {go1.cg6};
			\addlegendimage{legend image code/.code={\node[right] {$\phi\;\pm$};}};
			\legend{$0.535$, $0.555$, $0.575$, $0.003$};
			
			\nextgroupplot[%
				xlabel=$r/\sigma$, xmin=1.8, xmax=6,%
				ytick pos=right, ylabel near ticks,%
				ylabel={$\mathcal{G}_u(r,t^{dh}) / \Delta r^2(t^{dh})$}, ymin=1e-4, ymax=2e-1, ymode=log,%
				]
			\addplot+[mark=none, forget plot, domain=0:6] {0.41037249/x*exp(-x/1.46551291)};
			\addplot+[only marks] file{4446.gu};
			\addplot+[mark=none, forget plot, domain=0:6] {0.36214379/x*exp(-x/2.52390372)};
			\addplot+[only marks] file{5079.gu};
			\addplot+[mark=none, forget plot, domain=0:6] {0.41932273/x*exp(-x/3.991882)};
			\addplot+[only marks] file{go1.gu};
			\addlegendimage{legend image code/.code={\node[right] {$\phi\;\pm$};}};
			\legend{$0.535$, $0.555$, $0.575$, $0.003$};
			
		\end{groupplot}
	\end{tikzpicture}
	\begin{itemize}
		\item Ornstein-Zernike fit for both
		\begin{itemize}
			\item Indication of criticality
		\end{itemize}
		\item No icosahedra here, only crystal-like order
	\end{itemize}
\end{frame}

\begin{frame}{Size of the crystal-like ordered regions}
	\begin{textblock*}{0.6\textwidth}(10mm,88mm)
		\simplephasediagram{}
	\end{textblock*}
	\begin{columns}
	\column{0.45\textwidth}
	\resizebox{\columnwidth}{!}{\begin{LARGE}\input{fit_G6.tex}\end{LARGE}}\\
	\resizebox{\columnwidth}{!}{\begin{LARGE}\input{g6.tex}\end{LARGE}}
	\column{0.55\textwidth}
	\begin{itemize}
		\item With coarse-graining
		\begin{itemize}
			\item Ornstein-Zernike fit
			\[ G_6(r) \propto r^{-1}\exp( -\frac{r}{\xi_6} )\]
			\item Growing correlation length
		\end{itemize}
		
		\bigskip
		\item Without coarse-graining
		\begin{itemize}
			\item Alternatively positive and negative
			\item Susceptibility $\simeq 0$
			\item No long range correlation
		\end{itemize}
	\end{itemize}
	\end{columns}
\end{frame}

%\section{Crystallisation}
%
%\begin{frame}{Crystallisation at the wall}
%	\begin{columns}[T]
%	\column{0.5\textwidth}
%	\includegraphics[width=\columnwidth]{X_mountains_Q6}\\
%	\centering{Colour by degree of crystallisation ($Q_6$)}
%	\column{0.5\textwidth}
%	\includegraphics[width=\columnwidth]{X_mountains_W4}\\
%	\tikz\shade[ball color=blue] (0,0) circle (0.5em); FCC\quad
%	\tikz\shade[ball color=yellow] (0,0) circle (0.5em); HCP\quad
%	($W_4$)
%	\end{columns}
%	\begin{itemize}
%		\item Real crystals: $Q_6>0.4$
%		\item Not growing homogeneously
%	\end{itemize}
%\end{frame}
%
%\begin{frame}{Heterogeneous nucleation process}
%	\tikz\shade[ball color=green!50!black] (0,0) circle (0.5em); $0.25<Q_6<0.4$: crystal-like order\quad
%	\tikz\shade[ball color=red!50!black] (0,0) circle (0.5em); $0.4<Q_6$: crystal
%	\begin{columns}
%	\column{0.25\textwidth}
%	\includegraphics[width=\columnwidth]{X_t000}\\
%	$t=\unit{0}{\hour}$
%	
%	\bigskip\includegraphics[width=\columnwidth]{X_t100}\\
%	$t=\unit{20}{\hour}$
%	%
%	\column{0.25\textwidth}
%	\includegraphics[width=\columnwidth]{X_t025}\\
%	$t=\unit{5}{\hour}$
%	
%	\bigskip\includegraphics[width=\columnwidth]{X_t150}\\
%	$t=\unit{30}{\hour}$
%	%
%	\column{0.25\textwidth}
%	\includegraphics[width=\columnwidth]{X_t050}\\
%	$t=\unit{10}{\hour}$
%	
%	\bigskip\includegraphics[width=\columnwidth]{X_t200}\\
%	$t=\unit{40}{\hour}$
%	%
%	\column{0.25\textwidth}
%	\includegraphics[width=\columnwidth]{X_t075}\\
%	$t=\unit{15}{\hour}$
%	
%	\bigskip\includegraphics[width=\columnwidth]{X_t299}\\
%	$t=\unit{60}{\hour}$
%	\end{columns}
%\end{frame}
%
%\begin{frame}{Heterogeneous nucleation movie}
%	\begin{columns}
%	\column{0.6\textwidth}
%	%\movie[externalviewer, inline=false, text={
%		\includegraphics[width=\columnwidth]{X_t150}%}]{}{}{X.mov}
%		\\
%	\tikz\shade[ball color=green!50!black] (0,0) circle (0.5em); $0.25<Q_6<0.4$: crystal-like order\\
%	\tikz\shade[ball color=red!50!black] (0,0) circle (0.5em); $0.4<Q_6$: crystal
%	\column{0.4\textwidth}
%	\begin{itemize}
%		\item Bond order first
%		\item Nucleation from the ordered regions
%		\item $\simeq$ wetting
%	\end{itemize}
%	\end{columns}
%\end{frame}
%
%\begin{frame}{Density profiles}
%	\centering\resizebox{0.8\textwidth}{!}{\input{X_profile}}
%	\begin{itemize}
%		\item Layering at the wall
%		\item Icosahedra's centres are also layered
%	\end{itemize}
%\end{frame}
%
%\begin{frame}{Normalised density profiles}
%	\begin{columns}
%	\column{0.5\textwidth}
%	\resizebox{\columnwidth}{!}{\input{X_profile_normed}}
%	\column{0.5\textwidth}
%	\begin{itemize}
%		\item Normalise by the volume available to the icosahedron
%		\item Excluded from crystal-like order
%	\end{itemize}
%	\end{columns}
%	\begin{itemize}
%		\item Icosahedral order penetrate deep in the valley between the nucleus
%		\item Icosahedra fit better between the layers
%	\end{itemize}
%\end{frame}