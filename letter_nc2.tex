\documentclass[a4paper, rebuttal, parskip=true, firsthead=false, fromemail=true, foldmarks=false]{scrlttr2}
\usepackage{amsmath}
\usepackage{amsfonts}
\usepackage{amssymb}
\usepackage[british]{babel}
%\address{Mathieu Leocmach and Hajime Tanaka,\\ Institute of Industrial Science,\\ University of Tokyo}
%\signature{Mathieu Leocmach and Hajime Tanaka} 
\begin{document} 
\begin{letter}{Dr. Nicky Dean\\
Associate Editor\\
Nature Communications}
\opening{\bf Dear Nicky,}

Thank you very much for your e-mail concerning our manuscript (NCOMMS\nobreakdash-12\nobreakdash-00580\nobreakdash-T) together with the comments of the Reviewers. 

Following the constructive comments and suggestions of the Reviewers, we have revised our manuscript. 
We believe that we have been able to answer the comments of both reviewers on a satisfactory level and thus the revised manuscript has been much improved, thanks to the valuable comments of both Reviewers. We note that the parts revised to answer the Reviewers have been highlighted with blue characters in the revised manuscript.

At the same time, we have revised our manuscript to follow the format requirements of Nature Communication: the abstract has been reduced to 150 words, we added headings and subheading, a final paragraph was added to the (shortened) introduction. We also transferred most of the supplementary material to the method section, as you kindly suggested.


We hope that you and your reviewers would find that the revised manuscript is now suitable for publication in Nature Communications. 

\closing{\bf Sincerely yours,} 
\clearpage

\textsf{\textbf{Replies to the comments of Reviewer \#2}}

First, we thank the Reviewer for his or her commitment to the improvement of our manuscript. Following the kind and knowledgeable comments of the Reviewer, we have been able to make significant progresses.

\begin{quotationi}
The authors have substantially addressed the criticisms that were previously made, but a key point that arises from the latest modifications remains to be addressed.

The calculation of the dynamical length scale from fitting the four-point function (S5) with an OZ form (S7) relies on having a good estimate of $\chi_4(t)$ for the $q\rightarrow 0$ intercept. As the authors point out, this intercept would be extremely difficult to measure experimentally through the approach used in the simulations of SI Ref. 24. The alternate approach of directly measuring the $q\rightarrow 0$ limit of $S_4(q,t)$ is also more difficult here than in the simulations of SI Ref. 24, presumably because of the specifics of the experimental setup (although the experimental cell contains a lot more particle than the 80k used in simulations). This difficulty, however, is expected to have a larger impact on the $S_4$ results at high $\phi$ than at low $\phi$, because at high $\phi$: (1) the low $q$ regime becomes more important for the OZ (S7) fit and (2) the corrections to $\chi_4(t)$ become more significant (see SI Ref. 24 Fig. 8). It is therefore unclear why the authors claim that their $\phi<0.56$ results are less robust than the ones at $\phi>0.56$ (SI p.6). Fortunately, according to SI Ref. 24's Fig. 6, at high $\phi$ the real space $G_4(r,t)$ function should bypass both of these difficulties in estimating $\xi_4$. Because the core of the author's thesis relies on having a reliable estimate of the dynamical length scale and that the authors' results for that length scale are now remarkably smaller at high $\phi$ than those of SI Ref. 24's Fig. 11, such check is not superfluous. If the difference between simulation and experiment is sustained, a brief discussion of the difference would be in order. Note also that a figure like SI Ref. 24's Fig. 17 would be more instructive than the current Fig. S5.
\end{quotationi}

The absence of $q\rightarrow 0$ value of $S_4$ was indeed a weakness of our analysis. Rather than going back to real-space correlation function (the Reviewer previously stressed that ``describing the envelope around a noisy, slowly oscillating yet rapidly decaying function is numerically quite challenging'' and we could not agree more) we found a way to measure directly $\chi_4$ from our data.

The comment of the Reviewer reminded us of a discussion we had with Grzegorz Szamel about his paper (above SI Ref. 24). The main issue in measuring $\chi_4$ from the fluctuations of the number of overlapping particles $N_s$ in simulation is the lack of variation of the total number of particles $N$. This is a consequence of the ensemble of fixed $N$ used in simulations and its small size. However in our experimental window we do not have a fixed number of particles and furthermore our sample volume has contact with a large reservoir 
(the whole sample). Thus we can have the following relation:
\[
\chi_4(t) = (\langle N_s^2\rangle - \langle N_s\rangle^2) /  \langle N\rangle. 
\]
We may say that we have a very large system (the entire suspension in a sample tube) that reaches the thermodynamic limit where all ensembles become equivalent. Our experimental window is probing a small part of this large system. At a given time, we have access to a limited range of wave numbers. However a time average probes the whole large system state and is thus ensemble independent.

We use the value of $\chi_4$ given by the above formula as $S_4(q=0,t)$ in our fit. The figures were updated accordingly and Fig. S5 is now on the model of Ref. 24's Fig. 17. You may notice decrease at low $\phi$ and increase a high $\phi$, leading to a slightly larger slope. 

However, since the first report of the Reviewer, you drew our attention on the fit of correlation functions and we had the pleasure to learn a lot from this exchange. We thus looked closer to our structural correlation function ($G_6$). We found that the limit of this function at large $r$ was not null (contrary to periodic boundary condition case, e.g. Ref. [7]) as previously assumed and that this was affecting our fit. Subtraction of this baseline gave much more reliable data that we know report in the inset of Fig. 1.

Despite all these changes in the method, we still found that the dynamic correlation length $\xi_4$ behaves coherently with the static correlation length $\xi_6$ (see the inset of Fig. 1), 
with a slight difference (by a factor of 1.6) in the bare correlation length. This slight difference is acceptable considering the difference in the 
physical quantity (binary overlap function 
for dynamic one and continuous complex bond orientational order for static one) for which we calculate the spatial correlation.  

We argue that the difference in the value of the dynamical correlation length between ours and those in Ref. 24 has two possible causes:
\begin{enumerate}
\item In Ref. 24 the length are given in unit of the smaller diameter. In our system we use the mean diameter. This may account for a global factor up to $1.2$ (the mixture in Ref. 24 was a mixture of 50:50 particles with the size ratio of 1.4). This largely explains the discrepancy. At $\phi=0.5$ the value of $\xi_4$ in Ref. 24 is about 0.9, 
which corresponds to $0.9/1.2=0.75$ in our unit. This is very close to the value of our $\xi_4$ at $\phi=0.5$, about 0.7.   
\item The growth of the dynamical correlation length with decreasing $\phi$ in Ref. 24 is slightly more rapid than ours. 
This may be explained by a fundamental difference between systems: binary vs polydisperse, different fragilities, different underlying crystal structures. 
There is no basis to expect the same growth behaviour of the correlation length on noting these differences between the two systems. 
Here it may be worth noting that the results of our simulations of polydisperse ($6\%$) WCA spheres (Ref. [7]) are consistent with ours. 
\end{enumerate} 


\begin{quotationi}
Minor points:
\begin{itemize}
\item For completion, the authors should define explicitly $\tau_\alpha$ and $\chi_4(t)$.
\end{itemize}
\end{quotationi}
Thanks. We have now defined both terms.

\begin{quotationi}
~
\begin{itemize}
\item On p. 3 4th line, the authors probably mean to write ``Modern spin-glass-type theories of the structural glass transition''.
\end{itemize}
\end{quotationi}
Following the kind suggestion, we corrected this sentence.

We hope that the Reviewer would think that the revised manuscript has been much improved and is now suitable for publication in Nature Communications. 
 
\clearpage

\textsf{\textbf{Replies to the comments of Reviewer \#3}}

\begin{quotationi}
I will state one last confusion I have, and the authors can determine if they this is my problem alone or if this is something they can clarify. This regards the supplemental information and in particular Eqn. S1. These should be regarded as optional comments; the manuscript can be published as-is if the authors like.
\end{quotationi}

First, we thank the Reviewer for having carefully read our revised manuscript and provided useful comments to improve our manuscript. Hereafter we reply to the comments one by one.

\begin{quotationi}
First, I am still confused why the point spread function does not result in an offset. I am happy to consider the case of a symmetric, Gaussian point spread function, as the authors suggest. Shouldn't this result in a raw image with a larger-than-real particle size? For that matter, if the dye within the particle is some function of the radial distance within the particle, might not that influence the results? For example, if most of the dye was concentrated in the core of the particle? If the authors could explain this better that might be nice, at the very least they could do so in their future paper that promises to go into more detail on this method.
\end{quotationi}

Indeed the gaussian point spread function do result in a larger-than-real particle size. However, the real radii are proportional to the measured radii thus the problem reduces to the determination of a system-wise proportionality constant. This can be done by analysis of the point spread function or by measurement of the position of the first peak of $g(\hat{r})$ defined in Supplementary Information (that is what we have done here). One can also deconvolve the image prior to the tracking.

We have not thought about the dye repartition within a particle, probably because our colloids had a rather uniform repartition. We thank the Reviewer for this idea and we may discuss about that in our future paper on the subject.

\begin{quotationi}
Second, I do not understand how the right-hand side of Eqn. (S1) is n-independent, although I am willing to assume that the authors are correct. Here is my logic, and hopefully the authors can explain to me the flaw in my reasoning. One looks for the best sigma for a given particle. Given that the sigma's are chosen from a sequence of sigma's, I could say that the best sigma for a given particle is $\sigma_j = 2^{j/n} \sigma_0$, using the formula defining sigma in the text. Then the particle radius is $R = 2^{j/n} \sigma_0 \sqrt{3 \ln 2 / (n (1-2^{-2/n}))}$. For different choices of $n$, I understand that the ratio $j/n$ will be approximately the same but vary slightly depending on which $j$ is optimal. But how then does the $n$-dependence in the square root cancel out? I just can't see this mathematically. Perhaps there is some reason that the ratio $j/n$ does change when $n$ is changed? Similar to the first point, I am willing to trust that the authors have done the work correctly but I do not think I could reproduce their algorithm with my current level of understanding of what is written. And, the algorithm sounds straightforward enough and in general is clear, so I am frustrated by the gap in my understanding.
\end{quotationi}

The cancelling of the $n$ dependence is indeed not obvious in Eq. (S1) alone, so we have reformulated this paragraph.


The analytical response of a binary ball to a Gaussian blur, at the center of the ball is a function of dimensionless ratio $x=R/\sigma$: $G(x)$. The DoG response function is the difference between the two such functions. However, the choice of the width of the two functions is not arbitrary: we make the difference between two consecutive blurred images, the image blurred by $\sigma_{j+1}$ and the image blurred by $\sigma_j$. Therefore, 
\[
DoG(R,\sigma, \alpha) = DoG(x, \alpha) = G(x/\alpha) - G(x) \quad\text{with }\alpha=2^{1/n}.
\]
Here it is clear that minimizing $DoG(x, \alpha)$ with respect to $x$ yields a value $x^*$ that depends only on $\alpha$. This means the optimal value of $\sigma$ is proportional to $R$.  Both $\sigma^*$ and the proportionality constant are function of $n$, but $R$ is an absolute value independent of the way we measure it.

We hope that the Reviewer would think that the revised manuscript is now suitable for publication in Nature Communications. 


%\cc{Cclist} 
%\ps{adding a postscript} 
%\encl{list of enclosed material} 
\end{letter} 
\end{document}