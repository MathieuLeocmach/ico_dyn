\documentclass[a4paper, rebuttal, parskip=true, firsthead=false, fromemail=true, foldmarks=false]{scrlttr2}
\usepackage{amsmath}
\usepackage{amsfonts}
\usepackage{amssymb}
\usepackage[british]{babel}
%\address{Mathieu Leocmach and Hajime Tanaka,\\ Institute of Industrial Science,\\ University of Tokyo}
%\signature{Mathieu Leocmach and Hajime Tanaka} 
\begin{document} 
\begin{letter}{Dr. Nicky Dean\\
Associate Editor\\
Nature Communications}
\opening{\bf Dear Nicky,}

Thank you very much for your e-mail concerning our manuscript (NCOMMS\nobreakdash-12\nobreakdash-00580\nobreakdash-T) together with the comments of the Reviewers. 

Following the constructive comments and suggestions of the Reviewers, we have revised our manuscript. 
We believe that we have been able to answer the comments of both reviewers on a satisfactory level and thus the revised manuscript has been much improved, thanks to the valuable comments of both Reviewers. 

We note that revised parts have been highlighted with blue characters in the revised manuscript. 

We hope that you and your reviewers would find that the revised manuscript is now suitable for publication in Nature Communications. 

\closing{\bf Sincerely yours,} 
\clearpage

\textsf{\textbf{Replies to the comments of Reviewer \#2}}

First, we thank the Reviewer for having carefully read our revised manuscript and provided useful comments to improve our manuscript. 
Hereafter we reply to the comments one by one. 

\begin{quotationi}
1) (SI, p.3) The small crystal-like regions observed are claimed to be much smaller than the critical nucleus size, and can therefore be neglected. One may presume that this critical nucleus estimate is based on the polydisperse crystallization results of Auer and Frenkel [Nature 413, 771 (2001) - not cited in the text] for neighboring polydispersity 5\% and 8.5\%. A rough calculation based on these results and classical nucleation theory indeed suggests that the radius of the critical clusters should be between 2 and 3 particle diameters wide at the highest density studied, which is much larger than the crystal-like regions reported. The recent work of Schilling et al. [Phys. Rev. Lett. 105, 025701 (2010) - not cited in the text] suggests, however, that a description of nucleation beyond the classical scenario better captures the process of hard sphere crystallization at these relatively high densities. Schilling's low symmetry crystal regions (LSC), whose definition resembles the MRCO's, are observed to be at the origin of crystallization. As this manuscript's authors themselves recognize, their experimental system is studied under conditions where crystallization is kinetically accessible (crystallizing samples were discarded). These crystal-like regions thus cannot be assumed to be unrelated to early stages of crystallization. If the LSC-MRCO equivalence is correct, it suggests that a MRCO-based mechanism for a growing static length scale in glass formers is unlikely to be very general, as it relies on the system's partial crystallization.
\end{quotationi}

We included in the SI the estimation of the critical nucleus size suggested by the Reviewer and we now cite the paper by Auer and Frenkel.

We do agree that our \textsc{mrco} or the \textsc{lsc} of Schilling et al. are related to early stage of crystallisation. This is indeed the topic of some recent papers of our lab (see the newly added discussion in SI for references) covering hard spheres and gaussian-core model. 

However, we stress that crystal-like solid regions, which are identified by the method of Frenkel and his coworkers, are still smaller than the critical nucleus size, which is confirmed by the fact that they never grow but always decay. In other words, there were no crystal nucleation events in the course of our experiments. In relation to this, there is probably a misunderstanding about the samples that crystallised and that we discarded: in these cases the nucleation was heterogeneous (at the wall) and probably enhanced by flow alignment when introducing the suspension in the capillary. We never observed crystal nucleation and growth from the bulk. We have described this explicitly in the revised manuscript to avoid the misunderstanding, for which we were responsible. 

Concerning the \textsc{mrco}, on the other hand, regions of high \textsc{mrco} have a finite lifetime comparable to the structural relaxation time and should be regarded as  usual thermal fluctuations of the system, as common as the fast and slow regions of the dynamic heterogeneity. We confirm this observations in simulations of hard spheres and gaussian core model. Pedersen et al. observed the same behaviour in Wahnstr\"om binary Lennard-Jones (``crystal'' being here a Frank-Kasper phase and \textsc{mrco} being icosahedral). This has also been described in SI. 

We also note that crystal-like solid low-symmetry regions (\textsc{lsc}) is not equivalent to \textsc{mrco} and the former spontaneously appears inside \textsc{mrco} with some statistical probability only if the local density there is high enough as a consequence of thermal fluctuations. Thus, the former is always embedded in the latter, as can be seen in Supplementary Fig. S3. This has also been stated in SI.

Nevertheless we agree that our system is only weakly frustrated against crystallisation. However, hard sphere mixtures, gaussian core, and non additive Lennard-Jones mixtures are considered as a good approximation to some realistic systems (e.g. colloids, star polymers, metallic glasses). For this class of systems (weak-to-medium frustration against crystallisation), we expect our scenario to hold. We expect structures locally reminiscent of the symmetry of the crystal to be present as common fluctuations of the supercooled liquid. Their relative stability and ability to reach medium-range sizes makes them a good candidates to explain the slow regions of the dynamic heterogeneity.

Without further tests of our scenario in strongly frustrated systems, however, claiming generality was indeed overstated. We hope that the above discussion (included shortly in the conclusion of the main text and developed in SI) will lift your objection.


\begin{quotationi}
2) Yet even assuming that crystallization does not taint the measure of the growing the static correlation length, problems with the dynamical length scale extraction shed serious doubts on the subsequent analysis. The SI material does not report the exact definition of mobility used (Eq. S5), but it is probably similar to that used by Flenner, Zhang, and Szamel [PRE 83 (2011) - not cited in the text] for studying a similar four-point function in a binary mixture of hard spheres. Although the particle-size distributions of the two systems are different, one expects that at the lower-end of the density range studied (a density that is even below the 52\% packing fraction generally considered to be the onset of caging, and a density where the two fluids present essentially the same local order) a same dynamical length, i.e., at most one particle diameter, is expected. In Fig. 1b), however, the value reported is closer to 1.5, which suggests that the numerical/experimental error on that point is $\geq 50\%$. Such error is understandable because describing the envelope around a noisy, slowly oscillating yet rapidly decaying function is numerically quite challenging (a reciprocal space analysis could help - see Flenner et al. above). Assuming that the lowest-density data point is indeed close to 1.0, the quality of the fit for the dynamical length scale (described on p. 6) would then be much diminished. Additionally, the dynamical length scale would then grow by a factor of 4 over a density interval where the static length scale increases by at most 80\%. This decoupling of the two phenomena, which suggests a correlation rather than a causation between the two observables, is the opposite of the conclusion presented in the manuscript. Interestingly, this alternate conclusion is broadly in line with the analysis of Charbonneau et al [Phys. Rev. Lett (201)], with which the authors claim disagreement.
\end{quotationi}

We thank the Reviewer for having pointed us the differences between our method and the one used more generally to characterise the spatial extent of dynamical heterogeneity, in particular, for having caught our attention to the method of Flenner, Zhang, and Szamel. The analysis in Fourier space has indeed yielded clearer results for real scalar order parameter fields. However, as will be shown below, it was more than a fit issue. 

In the previous version of our paper we were correlating in space $\delta u_i(t) = \Delta r_i(t)-\langle\Delta r(t)\rangle$ the fluctuation of the norm of the displacement of the particles. This was indeed a rather archaic method used in the first studies of dynamical heterogeneity. Why was it abandoned? Probably because the distribution of $\delta u_i(t)$ is extremely skewed, thus the few very fast particles are given a very large weight in the correlation function. Therefore the resulting correlation length $\xi_u$ corresponds to the size of the fastest regions. This method was not coherent with the main topic of our paper, i.e. the study of the ordered (slow) regions, making the story less understandable. Actually the previous version of the inset of Figure 1 was showing two length scales ($\xi_6$ structural and $\xi_u$ dynamical) with the same volume fraction dependence, but with very different absolute values (prefactors). As pointed out by the Reviewer, this made our point, the coupling between structure and dynamics, less convincing.

The state of the art method, kindly indicated by Reviewer \#2, relies on an overlap function that distinguishes in a binary way between fast and slow particles. More precisely, the overlap function reduces the original system into a system where only slow particles are left. The correlation length $\xi_4$ extracted is thus the size of the slow regions, on which our interest focusses. We discovered that $\xi_4$ was systematically smaller than $\xi_u$ by a factor of $\approx 2$, leading to values very close to our structural correlation length $\xi_6$. It was not the sole lowest density point that was overestimated by our previous method but the whole density range.

We have rewritten part of our paper and redrawn Figure 1 around the correlation length $\xi_4$. The former $\xi_u$ is now discussed solely in SI, including a discussion about the prefactor difference. We think it is due to the respective dimensionality of the fast (in general elongated) and slow (compact in our system) regions.
Thanks to the comment of the Reviewer, this issue has been significantly improved and now the behaviour of the dynamical correlation length is very much consistent with that 
of the static one. 

\begin{quotationi}
3) Finally, the previous reviews missed the following point. Although clarifying Ref. 3's description of a spin-glass-type theory of the glass transition is a useful aim, Ref. 3's treatment should not be considered as "the" spin-glass type theory for structural glasses. More modern spin-glass-type descriptions for a growing static length indeed make no direct reference to growing icosahedral order, e.g., Biroli et al. Nature Physics (2008), Parisi and Zamponi, Rev. Mod. Phys. (2010). The analysis therefore does not allow one to conclude that the spin-glass nor the RFOT approaches are generally at odds with their results.
\end{quotationi}

We have amended our introduction along the line suggested by the Reviewer, citing the above references. The use of ``the'' in this context was indeed inappropriate.

\begin{quotationi}
Small corrections:
\begin{itemize}
\item The first author of SI-Ref. 9 should be listed as ten Wolde, P. R.
\item The values for the fitted $\phi_0$ and the fragility index D should be reported.
\item The definition of $\delta u$ (Eq. S5) should be reported.
\item The binary mixture system studied by SI-Ref. 13 does indeed have a "symmetry-broken" (crystal) phase with six-fold symmetry (FCC or HCP), although that phase does require demixing [see Hopkins, Stillinger, and Torquato Phys. Rev. E 85, 021130 (2012) and Phys. Rev. Lett (2011) - not cited in the text].
\end{itemize}
\end{quotationi}

We corrected each of theses errors. Thank you for your attention to details. In particular we cited Hopkins et al. in SI.

We hope that the Reviewer would think that the revised manuscript has been much improved and is now suitable for publication in Nature Communications. 
 
\clearpage

\textsf{\textbf{Replies to the comments of Reviewer \#3}}

First, we thank the Reviewer for having carefully read our revised manuscript and provided useful comments to improve our manuscript. Hereafter we reply to the comments one by one. 

\begin{quotationi}
The revised manuscript has addressed my concerns. However, having seen the supplemental material, I now have some additional concerns (which are all small). I do not need to see the manuscript again, although I would be happy to re-review it if the editors wish. 

1. The value of $\phi_0$ should be given (page 6, the fit to Fig. 1). 
\end{quotationi}
Thank you. The value of $\phi_0$ has now been given. 

\begin{quotationi}
2. There are some minor typos: prefect (page 6) should be perfect, nosier (page 10) should be noisier, eatimated (SI page 5) should be estimated.
\end{quotationi}
Thank you. Typos were corrected.

\begin{quotationi}
3. In the supplemental materials, the authors should show a raw image from their experiments. This should be done so that one can judge the quality of the imaging, but also equally important, it would be a simple way to present information about the sample. We might be able to see the polydispersity, for example, to get a sense of it to match the data of Fig. S2a.
\end{quotationi}
We have followed the suggestion of the Reviewer and now showed raw images in Fig.~S2 (details of XY and XZ slices). 

\begin{quotationi}
4. Eqn. S1 uses sigma. It should be clarified that this is the sigma of the best fit, at least, I think that's what this means. Although given that sigma, I am not sure why 'n' appears in Eqn. S1. In general I am confused by Eqn. S1 and not quite sure how it is used. Also, I wonder about the statement that $\sigma$ is proportional to $R$. Is this strictly true? Wouldn't the point spread function broaden out sigma? I would have expected a linear relationship, $\sigma = a\times R + b$, with $b$ related to the point spread function. The word "proportional" implies $b=0$.
\end{quotationi}

We have added the word ``optimal'' to stress that the $\sigma$ of Eq.~(S1) is indeed the value of the scale where our response function is minimum. Because our response function includes $n$ in its definition, the scale of $\sigma$ is not absolute but depends on $n$. However the influence of the choice of $n$ cancels out among the different factors of the right-hand-side of Eq.~(S1) to yield an $R$ that is parameter-free. The errors on $R$ do depend on the number of subdivisions.

An arbitrary point-spread-function may indeed add a $b$. However, when considering a Gaussian point-spread-function we have $b=0$. Most of the fast deconvolution methods approximate point-spread functions to a Gaussian. While this approximation holds, we believe that $b$ is indeed negligible.


\begin{quotationi}
5. On page 2, it says that if the relative brightness of the particles isn't accounted for, then it changes the size. Reading the text, I have the implication that the relative brightness is not taken into account. Is that correct? What amount of error does this introduce for the particles in this experiment? I assume it is not a big problem, but this should be explained.
\end{quotationi}

Our formulation was indeed difficult to understand. We clarified saying that:
\begin{itemize}
\item If we do not take the brightness into account, (small) errors may occur
\item The brightness data can be extracted from the values of the DoG at the centre of the particles.
\item We \emph{do} use this procedure to extract our data.
\end{itemize}

\begin{quotationi}
6. Figure S1 is hard to understand. What is the difference between a and b? Which method (a or b) is used for these experiments? Are the 2D circles the same in all panels (a-h); is the only difference the grey spheres? Why are some of the spheres missing in b? One solution might be to omit (b) entirely, if the authors have only used (a) for the results in the main text. Otherwise, all of this needs to be explained.
\end{quotationi}

We removed (b) entirely as suggested and added a 3D representation of the raw image to allow direct comparison. We also clarified the caption of the figure. The 2D circles are indeed identical and act as a guide for the eye. They are the results of our algorithm applied in each 2D slice of the 3D picture. We found that in 2D our code was placing and sizing the bright disks almost as a human would do, so these circles are a good reference of where the particles might be. However, automatic reconstruction of the 3D positions of the particles from piles of 2D circles proved to be challenging. The direct 3D localisation and sizing is much more efficient and precise, although challenging for a human, especially in a dense environment. This was the point of the former (b), but as you kindly pointed this cannot be shown without a full explanation.


We hope that the Reviewer would think that the revised manuscript is now suitable for publication in Nature Communications. 


%\cc{Cclist} 
%\ps{adding a postscript} 
%\encl{list of enclosed material} 
\end{letter} 
\end{document}