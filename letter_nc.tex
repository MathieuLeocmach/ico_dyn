\documentclass[10pt,a4paper]{letter}
\usepackage[utf8x]{inputenc} \usepackage{ucs}
\usepackage{amsmath}
\usepackage{amsfonts}
\usepackage{amssymb}
\address{Mathieu Leocmach and Hajime Tanaka,\\ Institute of Industrial Science,\\ University of Tokyo}
\signature{Hajime Tanaka} 
\begin{document} 
\begin{letter}{}
\opening{Dear editor,\\Dear reviewers,} 
 
We respond mainly to the concerns of Reviewer \#2, quoted below.

\begin{quotation}
1) (SI, p.3) The small crystal-like regions observed are claimed to be much smaller than the critical nucleus size, and can therefore be neglected. One may presume that this critical nucleus estimate is based on the polydisperse crystallization results of Auer and Frenkel [Nature 413, 771 (2001) - not cited in the text] for neighboring polydispersity 5\% and 8.5\%. A rough calculation based on these results and classical nucleation theory indeed suggests that the radius of the critical clusters should be between 2 and 3 particle diameters wide at the highest density studied, which is much larger than the crystal-like regions reported. The recent work of Schilling et al. [Phys. Rev. Lett. 105, 025701 (2010) - not cited in the text] suggests, however, that a description of nucleation beyond the classical scenario better captures the process of hard sphere crystallization at these relatively high densities. Schilling's low symmetry crystal regions (LSC), whose definition resembles the MRCO's, are observed to be at the origin of crystallization. As this manuscript's authors themselves recognize, their experimental system is studied under conditions where crystallization is kinetically accessible (crystallizing samples were discarded). These crystal-like regions thus cannot be assumed to be unrelated to early stages of crystallization. If the LSC-MRCO equivalence is correct, it suggests that a MRCO-based mechanism for a growing static length scale in glass formers is unlikely to be very general, as it relies on the system's partial crystallization.
\end{quotation}

We think that we basically agree on all your arguments (we now cite Auer and Frenkel as well as Schilling et al.), however not on the final conclusion. 

Indeed it has been shown that fluid regions with a structure reminiscent of the crystal are precursors to crystallisation in hard spheres, gaussian-core model and probably Wahnstr\"om binary Lennard-Jones (see the newly added discussion in SI for references). However in these systems the presence of precursors to crystallisation do not imply the formation of a critical nucleus and growth. Furthermore, some precursors are always present (transiently) in the system sizes routinely used to characterise the supercooled liquid: \textsc{mrco} are usual fluctuations of the system, as common as the fast and slow regions of the dynamic heterogeneity. This is indeed one of the main points of our paper.

Nevertheless, we agree that our conclusions have to be nuanced in the absence of studies in other systems where crystallisation is less accessible, more strongly frustrated. We added a paragraph to discuss the generality of our scenario. We hope this will lift your objection.

\begin{quotation}
2) Yet even assuming that crystallization does not taint the measure of the growing the static correlation length, problems with the dynamical length scale extraction shed serious doubts on the subsequent analysis. The SI material does not report the exact definition of mobility used (Eq. S5), but it is probably similar to that used by Flenner, Zhang, and Szamel [PRE 83 (2011) - not cited in the text] for studying a similar four-point function in a binary mixture of hard spheres. Although the particle-size distributions of the two systems are different, one expects that at the lower-end of the density range studied (a density that is even below the 52\% packing fraction generally considered to be the onset of caging, and a density where the two fluids present essentially the same local order) a same dynamical length, i.e., at most one particle diameter, is expected. In Fig. 1b), however, the value reported is closer to 1.5, which suggests that the numerical/experimental error on that point is $\geq 50\%$. Such error is understandable because describing the envelope around a noisy, slowly oscillating yet rapidly decaying function is numerically quite challenging (a reciprocal space analysis could help - see Flenner et al. above). Assuming that the lowest-density data point is indeed close to 1.0, the quality of the fit for the dynamical length scale (described on p. 6) would then be much diminished. Additionally, the dynamical length scale would then grow by a factor of 4 over a density interval where the static length scale increases by at most 80\%. This decoupling of the two phenomena, which suggests a correlation rather than a causation between the two observables, is the opposite of the conclusion presented in the manuscript. Interestingly, this alternate conclusion is broadly in line with the analysis of Charbonneau et al [Phys. Rev. Lett (201)], with which the authors claim disagreement.
\end{quotation}

This part that was the longest to answer. We thank both reviewers for pointing us the differences between our method and the one used more generally to characterise the spatial extent of dynamical heterogeneity. We thank peculiarly Reviewer \#2 for pointing to us the method of Flenner, Zhang, and Szamel. The analysis in Fourier space yields indeed clearer results for real scalar order parameter fields. However as you will see it was more than a fit issue.

In the previous version of our paper we were correlating in space $\delta u_i(t) = \Delta r_i(t)-\langle\Delta r(t)\rangle$ the fluctuation of the norm of the displacement of the particles. This was indeed a rather archaic method used in the first studies of dynamical heterogeneity. Why was it abandoned ? Probably because the distribution of $\delta u_i(t)$ is extremely skewed, thus the few very fast particles are given a very large weight in the correlation function. Therefore the resulting correlation length $\xi_u$ corresponds to the size of the fast regions. This method was not coherent with the main topic of our paper, i.e. the study of the ordered (slow) regions. Actually the previous version of the inset of Figure 1 was showing two length scales ($\xi_6$ structural and $\xi_u$ dynamical) with the same volume fraction dependence, but with very different absolute values (prefactors). This made our point, the coupling between structure and dynamics, less convincing.

The state of the art method, kindly indicated by Reviewer \#2, relies on an overlap function that distinguishes in a binary way between fast and slow particles. More precisely, the overlap function reduces the original system into a system where only slow particles are left. The correlation length $\xi_4$ extracted is thus the size of the slow regions, on which our interest focusses. We discovered that $\xi_4$ was systematically smaller than $\xi_u$ by a factor of $\approx 2$, leading to values very close to our structural correlation length $\xi_6$. It was not the sole lowest density point that was overestimated by our previous method but the whole density range.

We rewrote part of our paper and redraw Figure 1 around the correlation length $\xi_4$. The former $\xi_u$ is now discussed solely in supplementary information, including a discussion about the prefactor difference. We think it is due to the respective dimensionality of the fast (in general elongated) and slow (compact in our system) regions.

\begin{quotation}
3) Finally, the previous reviews missed the following point. Although clarifying Ref. 3's description of a spin-glass-type theory of the glass transition is a useful aim, Ref. 3's treatment should not be considered as "the" spin-glass type theory for structural glasses. More modern spin-glass-type descriptions for a growing static length indeed make no direct reference to growing icosahedral order, e.g., Biroli et al. Nature Physics (2008), Parisi and Zamponi, Rev. Mod. Phys. (2010). The analysis therefore does not allow one to conclude that the spin-glass nor the RFOT approaches are generally at odds with their results.
\end{quotation}

We addressed this point, as well as the minor points raised by both Reviewers.
 
\closing{Sincerely yours,} 
%\cc{Cclist} 
%\ps{adding a postscript} 
%\encl{list of enclosed material} 
\end{letter} 
\end{document}