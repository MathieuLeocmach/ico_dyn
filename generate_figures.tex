\input{preamble}

\usepackage{pgfplots}
\usepgfplotslibrary{external}
\usepgfplotslibrary{groupplots}
\usetikzlibrary{positioning}
\usetikzlibrary{plotmarks}
\usetikzlibrary{matrix}
\usetikzlibrary{shadows}
\tikzexternalize
%\tikzset{external/force remake}
\tikzset{every mark/.append style={scale=0.8}}
\pgfplotsset{every axis/.append style={small}}

\begin{document}
\tikzset{external/force remake}
\begin{figure}
	\centering
	\begin{tikzpicture}[
		lab/.style={above right, text height=0.8em, text depth=0.2em, font=\Large\bfseries}%
		]%
		\pgfplotsset{cycle list name=black white}
		\pgfplotsset{fitc/.style={solid, no markers, forget plot, domain=1:1e4}}
		\begin{axis}[%
			anchor=right of south east,%
			width=0.45\textwidth, height=0.33\textwidth, %
			xlabel=$t/\tau_B$, xmode=log,%
			ylabel={Self ISF}, ymin=0,%
			legend columns=3, reverse legend, legend style={font=\tiny, at={(1,1.03), nodes={anchor=east}}, anchor=south east}, %
			]
			\addplot+[only marks] file {LS3954.isf};
			\addplot+[only marks] file {LS4446.isf};
			\addplot+[only marks] file {LS4582.isf};
			\addplot+[only marks] file {LS5079.isf};
			\addplot+[only marks] file {go1.isf};
			\addplot+[fitc] {0.943 * exp(-(x/4.60)^0.72)};
			\addplot+[fitc] {0.901 * exp(-(x/10.78)^0.72)};
			\addplot+[fitc] {0.894 * exp(-(x/15.51)^0.70)};
			\addplot+[fitc] {0.876 * exp(-(x/60.18)^0.60)};
			\addplot+[fitc] {0.787 * exp(-(x/2408.)^0.70)};
			\addlegendimage{legend image code/.code={\node[right] {$\phi\;\pm$};}};
			\legend{$0.497$, $0.535$, $0.540$, $0.555$, $0.575$, $0.003$};
			\node [lab] at (rel axis cs:0,0) {a};
		\end{axis}
		\begin{axis}[%
			anchor=left of south west,%
			width=0.45\textwidth, %
			xlabel=$\phi$,%
			ylabel=$\tau_\alpha/\tau_B$, ymode=log, %
			ytick pos=right, ylabel near ticks,%
			legend pos=south east,%
			cycle list={{black, mark=*},{black, mark=square}},%
			]
			%\addplot+[mark=none, forget plot, domain=0.49:0.58] {0.7874243*exp(0.32826088*x/(0.60063652-x))};
			\addplot+[mark=none, forget plot, domain=0.49:0.58] {exp(0.28970401*x/(0.59841615-x))};
			\addplot+[only marks] table[x index=0, y index=1]{xi_phi.dat};
			\addplot+[only marks] table[x index=0, y index=4]{xi_phi.dat};
			\legend{$\tau_\alpha$, $t^{dh}$};
			\node at (rel axis cs:0,1) (a) {};
			\node [lab] at (rel axis cs:0,0) {b};
		\end{axis}
		\begin{axis}[%
			tiny, at=(a), anchor=north west, %
			width=0.3\textwidth, %
			xlabel=$\phi$, xmin=0.49, xmax=0.58, xlabel near ticks,%
			xtick={0.5,0.52,...,0.6},%
			ylabel=$\xi/\xi_0$, ymax=10,
			ytick={2,4,6,8},
			ytick pos=right, ylabel near ticks,%
			label style={font=\tiny}, %
			legend pos=north west,%
			axis background/.style={fill=white},%
			]
			\addplot+[only marks, fill=black, %
				error bars/.cd, y dir=both, y explicit relative,%
				] table[x index=0, y expr=\thisrowno{3}/0.126, y error index=6]{scale.xi};
			\addplot+[mark=none, forget plot, domain=0.49:0.58] {(0.6/x-1)^(-2.0/3.0)};
			\addplot+[only marks, mark=square, %
				error bars/.cd, y dir=both, y explicit relative,%
				] table[x index=0, y expr=\thisrowno{5}/0.216, y error index=7]{scale.xi};
			\legend{$\xi_6$ structural, $\xi_4$ dynamical};
		\end{axis}
	\end{tikzpicture}
	\caption{\textbf{Dynamics of the system.} {\bf a,} Decay of the self-intermediate scattering function computed from the positions for several volume fractions. Lines are stretched exponential fit from which is extracted the structural relaxation time $\tau_\alpha$ {\bf b,} $\tau_\alpha$ and the characteristic time of the dynamic heterogeneity $t^{dh}$ scaled by the Brownian time $\tau_B$ as a function of $\phi$. The solid curve is the VFT fit of $\tau_\alpha$ ($\phi_0=0.600$, $D=0.328$). Inset: Dynamical ($\xi_4$) and structural correlation length ($\xi_6$) scaled by their respective bare correlation length, respectively $\xi_{4,0}=0.206\sigma$ and $\xi_{6,0}=0.126\sigma$. The curve is the power-law given in the text.}
	\label{fig:vft}
\end{figure}
\tikzset{external/force remake=false}
\begin{figure}
	\begin{tikzpicture}[
		lab/.style={below right, text height=0.8em, text depth=0.2em, font=\Large\bfseries}%
		]%
		\pgfplotsset{
			extra tick style={grid=major, major grid style={
				copy shadow={very thick, shadow xshift=0, shadow yshift=0, white, semitransparent}, 
				black}},%
			every axis/.append style={%
				height=0.4\textwidth,
				ymin=0,ymax=0.6,%
				extra y ticks={\Qstar}, extra y tick labels={},%
				enlargelimits=false,axis on top,
				colormap={bw}{gray(0cm)=(1); gray(1cm)=(1); gray(10cm)=(0)},%
				colorbar sampled,%
				},%
		}
		\begin{groupplot}[
			group style={
				group size=2 by 1,%
				yticklabels at=edge left,%
				horizontal sep=0pt,%
				},%
			anchor=below south west,%
			width=0.45\textwidth,%
			xtick scale label code/.code={},%
			colorbar horizontal, colorbar style={%
				samples=6, xtick={ 0.20,0.4,0.6, 0.8},% 
				extra y ticks={},%
				/pgfplots/colorbar shift/.style={yshift=0.3cm},
				at={(parent axis.north)}, anchor=below south, width=0.9*\pgfkeysvalueof{/pgfplots/parent axis width},
				xticklabel pos=upper,%
				label style={font=\footnotesize},
				},%
			]
		\nextgroupplot[%
			ylabel={$Q_6$}, xlabel=$Q_4$,%
			xmin=0,xmax=0.22,%
			colorbar style={%
				xlabel={per units of $Q_4\cdot Q_6$},% 
				xticklabels={$10^{1}$, $10^{2}$, $10^{3}$, $10^{4}$},%
				},%
			xticklabel={$\pgfmathprintnumber[fixed,precision=2]{\tick}$}
			]
		\addplot graphics
		[xmin=0,xmax=0.2,ymin=0,ymax=0.6]
		{Q4Q6go1};
		\node [lab] at (rel axis cs:0,1) {a};
		\node [below] at (axis cs:0.1909, 0.5745) {\textsc{fcc}};
		\node [below] at (axis cs:0.0972222, 0.484762) {\textsc{hcp}};
		\node [below] at (axis cs:0.0363696, 0.510688) {\textsc{bcc}};
		\draw[->, white,thick] (axis cs:0.05, 0.15) to [out=60, in=220] (axis cs:0.125, 0.4);
		
		\nextgroupplot[%
			xlabel=$10^3 \cdot w_4$, %
			xmin=-0.002,xmax=0.002,%
			xtickmin=-0.0015,%
			extra x ticks=0, extra x tick labels={},
			colorbar style={%
				xlabel={per units of $w_4\cdot Q_6$},% 
				xticklabels={$10^{3}$, $10^{4}$, $10^{5}$, $10^{6}$},%
				},%
			]
		\addplot graphics
		[xmin=-0.002,xmax=0.002,ymin=0,ymax=0.6]
		{w4Q6go1};
		\node [lab] at (rel axis cs:0,1) {b};
		\node [below] at (axis cs:-0.00067221, 0.5745) {\textsc{fcc}};
		\node [below] at (axis cs:7.47E-05, 0.484762) {\textsc{hcp}};
		%\node [below] at (axis cs:-0.01274716, 0.510688) {\textsc{bcc}};
		\node [above] at (axis cs:0.0015, \Qstar) {\footnotesize{$Q_6^*$}};
		\draw[->, white,thick] (axis cs:0, 0.15) to [out=120, in=280] (axis cs:-0.0005, 0.4);
		\draw[->, white,thick] (axis cs:0, 0.15) to (axis cs:7E-05, 0.35);
		
		\end{groupplot}
		
		\begin{groupplot}[
			group style={
				group size=2 by 1,%
				yticklabels at=edge left,%
				horizontal sep=0pt,%
				},%
			anchor=above north west,%
			width=0.38\textwidth,%
			xlabel=$10^2 \cdot w_6$,%
			xmin=-0.052,xmax=0.01,%
			extra x ticks={\wstar, -0.00782},%
			extra x tick labels={,},%
			xtick scale label code/.code={},%
			colorbar right, colorbar style={%
				samples=6, ytick={ 0.20,0.4,0.6, 0.8},% 
				yticklabels={$10^{1}$, $10^{2}$, $10^{3}$, $10^{4}$},%
				ylabel={per units of $w_6\cdot Q_6$},%
				extra x ticks={},%
				extra y ticks={},%
				label style={font=\footnotesize},
				},%
			]
		
		\nextgroupplot[ylabel={$Q_6$}, xtickmax=0,]
		\addplot graphics
		[xmin=-0.052,xmax=0.052,ymin=0,ymax=0.6]
		{u6Q6go1_scale};
		\node [lab] at (rel axis cs:0,1) {c};
		\node at (axis cs:\wstar,0.6) (a) {};
		\node at (axis cs:-0.00782,0.6) (b) {};
		\node [below] at (axis cs:-0.0026, 0.5745) {\textsc{fcc}};
		\node [below, right] at (axis cs:-0.052, 0.05) {\textsc{Ico}};
		\draw[->, white,thick] (axis cs:-0.001, 0.15) to [out=90, in=275] (axis cs:-0.0015, 0.33);
		\draw[->, white,thick] (axis cs:-0.001, 0.15) to [out=180, in=30] (axis cs:-0.025, 0.12);
		
		\nextgroupplot
		\addplot graphics
		[xmin=-0.052,xmax=0.052,ymin=0,ymax=0.6]
		{u6Q6phi3954_scale};
		\node [lab] at (rel axis cs:0,1) {d};
		\node [above] at (axis cs:-0.04, \Qstar) {\footnotesize{$Q_6^*$}};
		%\node [anchor=north east] at (rel axis cs:1, 1) {\footnotesize{$\phi = 0.497 \pm 0.03$}};
		\node [left] at (axis cs:\wstar,0.4) {\footnotesize $w_6^*$};
		\node [right] at (axis cs:-0.00782,0.4) {\footnotesize $w_6^{dod}$};
		
		\end{groupplot}
	\end{tikzpicture}
	\caption{Population of local structures function of \textsc{boo}. (a-c) For our deepest supercooled sample ($\phi=0.575\pm 0.03$) in the $(Q_4,Q_6)$-plane (a), $(w_4,Q_6)$-plane (b) and $(w_6,Q_6)$-plane (c). (d) is the same as (c) but for a liquid near the freezing point ($\phi = 0.497 \pm 0.03$). Colour represent the probability to find the structure (log scale). The arrows stress the ordering tendencies: the tendency toward \textsc{fcc} is always visible, but a tendency toward \textsc{hcp} can be distinguished in (b) and the tendency toward icosahedral order is visible in (c).}
	\label{fig:maps}
\end{figure}
\tikzset{external/force remake=false}

\begin{figure}
	\begin{tikzpicture}[
		lab/.style={anchor=north west, text height=0.8em, text depth=0.2em, font=\Large\bfseries}%
		]%
		\pgfplotsset{extra tick style={%
			grid=major, major grid style={%
				copy shadow={very thick, shadow xshift=0, shadow yshift=0, white, semitransparent}, %
				black}},}
		\begin{axis}[%
			name=twod,%
			at={(0, 0.425\textwidth)},
			width=0.5\textwidth,%
			height=0.5\textwidth,%
			ylabel=$Q_6$,%
			ymin=0,ymax=0.6,%
			xlabel=$10^2 \cdot w_6$,%
			xmin=-0.052,xmax=0.01,%
			xtick scale label code/.code={},%
			enlargelimits=false,axis on top,
			colormap={sd}{color(0cm)=(black) rgb(1cm)=(0.5, 0, 0) rgb(2cm)=(1, 0.5, 0) color(3cm)=(yellow) rgb(4cm)=(0.5, 0.75, 1) rgb(5cm)=(0.5, 0.75, 1)},%
			extra x ticks={\wstar, -0.00782},%
			extra x tick style={grid=major,	tick label style={anchor=south west}},%
			extra x tick labels={,},%
			extra y ticks={\Qstar},
			extra y tick labels={},%
			colorbar sampled, colorbar style={%
				samples=5, ytick={ 0, 0.25, 0.5, 0.75},% 
				yticklabels={$0$, $0.5$, $1$, $1.5$},%
				ylabel=${\Delta r^2(w_6, Q_6, t^{dh})}/{\langle\Delta r^2(t^{dh}) \rangle}$,%
				extra x ticks={},%
				yticklabel pos=right,%
				label style={font=\footnotesize},
				},%
			]
			\addplot graphics [xmin=-0.052,xmax=0.052,ymin=0,ymax=0.6]{sd_u6Q6_go1_color};
			\node [lab] at (rel axis cs:0,1) {a};
			\node [above] at (axis cs:-0.04, \Qstar) {\footnotesize{$Q_6^*$}};
			\node [anchor=west] at (rel axis cs:0, 0.8) {\footnotesize{$\phi = 0.575 \pm 0.03$}};
			\node [left] at (axis cs:\wstar,0.55) {\footnotesize $w_6^*$};
			\node [left] at (axis cs:-0.00782,0.55) {\footnotesize $w_6^{dod}$};
			\node [below] at (axis cs:-0.0026, 0.5745) {\textsc{fcc}};
			\node [below, right] at (axis cs:-0.052, 0.05) {\textsc{Ico}};
		\end{axis}
		\pgfplotsset{every axis plot/.append style={only marks}}
		\begin{groupplot}[%
			group style={
				group size=2 by 2,%
				yticklabels at=edge left,%
				horizontal sep=0pt,%
				},%
			width=0.5\textwidth,%
			ymin=0, ymax=1.5,%
			extra x tick style={grid=major,	tick label style={anchor=south west}},%
			extra y ticks={1},%
			extra y tick style={grid=major,	tick label style={anchor=south west}}, extra y tick labels={},%
			%legend columns=5,%
			legend style={
				font=\footnotesize,
				%at={(1,1)}, anchor=south,%
				at={(1,1)}, anchor=north east
				},%
			]
			\nextgroupplot[%
				xlabel=$10^2 \cdot w_6$, xmin=-0.0521089193, xmax=0.01,%
				xtickmax={0},%
				xtick scale label code/.code={},%
				ylabel=${\Delta r^2(w_6,Q_6,t^{dh})}/{\langle\Delta r^2(t^{dh}) \rangle}$,
				extra x ticks={\wstar, -0.00782}, extra x tick labels={$w_6^*$,$w_6^{dod}$},%
				extra y tick labels={bulk}
				]
			
			%\addplot table[x index=0, y expr={\thisrowno{1}/\thisrowno{2}}]{sd_nb_u6_phi3954.txt};
			\addplot table[x index=0, y expr={\thisrowno{1}/\thisrowno{2}}]{sd_nb_u6_phi4446.txt};
			\addplot table[x index=0, y expr={\thisrowno{1}/\thisrowno{2}}]{sd_nb_u6_phi5079.txt};
			\addplot table[x index=0, y expr={\thisrowno{1}/\thisrowno{2}}]{sd_nb_u6_go1.txt};
			\addplot+[domain=-0.0521089193:0.01, sharp plot, no markers] {9.34701 * x + 1.03764};
			\draw[->, thick] (rel axis cs:0.15, 0.1) to (rel axis cs:0, 0.1);
			\node[right] at (rel axis cs:0.15, 0.1) {\footnotesize Icosahedron};
			\node [lab] at (rel axis cs:0,1) {b};
			
			\nextgroupplot[%
				xlabel=$Q_6$, xmin=0, xmax=0.5745,%
				extra x ticks={\Qstar}, extra x tick labels={$Q_6^*$},%				
				]
			\addlegendimage{legend image code/.code={\node[right] {$\phi\;\pm$};}};
			\legend{$0.003$,%$0.497$, 
				$0.535$, $0.555$, $0.575$};
			%\addplot table[x index=0, y expr={\thisrowno{1}/\thisrowno{2}}]{sd_nb_Q6_3954.txt};
			\addplot+[domain=0.05:0.3, sharp plot, no markers, forget plot] {-0.76 * x + 1.1};
			\addplot table[x index=0, y expr={\thisrowno{1}/\thisrowno{2}}]{sd_nb_Q6_4446.txt};
			%\addplot+[domain=0.05:0.3, sharp plot, no markers, forget plot] {1.09-1.18*x+8.95*x^2-38*x^3};
			\addplot+[smooth, no markers, forget plot] coordinates {(0.08, 1.04) (0.2, 0.92) (0.32, .62)};
			\addplot table[x index=0, y expr={\thisrowno{1}/\thisrowno{2}}]{sd_nb_Q6_5079.txt};
			\addplot+[smooth, no markers, forget plot] coordinates {(0.08, 1.23) (0.1, 1.22) (0.15, 1.07) (0.25, 0.65) (0.35, 0.33) (0.45, 0.18) (0.5, 0.15)};
			\addplot table[x index=0, y expr={\thisrowno{1}/\thisrowno{2}}]{sd_nb_Q6_go1.txt};
			\draw[->, thick] (rel axis cs:0.8, 0.45) to (rel axis cs:1, 0.45);
			\node[left] at (rel axis cs:0.8, 0.45) {\footnotesize \textsc{fcc}};
			\node [lab] at (rel axis cs:0,1) {c};
		\end{groupplot}
		\path (twod.outer south) ++(0.525\textwidth, 0.6em) 
			node[anchor=south] {$w_6=w_6^*$} 
			node[inner sep=0, above=0] (one_ico) {\includegraphics[width=0.22\textwidth]{one_ico.png}};
		\node at(one_ico.south){$w_6=w_6^*$};
		\path (twod.outer north) ++(0.525\textwidth, -0.6em)
			node[inner sep=0, below=0] (one_mrco) {\includegraphics[width=0.22\textwidth]{one_mrco.png}};
		\node at (one_mrco.south) {$Q_6=Q_6^*$};
		\node [lab] at (one_mrco.north west) {d};
		\node [lab] at (one_ico.north west) {e};
	\end{tikzpicture}
	\caption{\textbf{Bond order mobility.} {\bf a,} Normalised mobility in the $(w_6, Q_6)$-plane for our most deeply supercooled sample. The colour scale is saturated at $1.5$ times the bulk mean square displacement. {\bf b-c,} Normalised mobility for icosahedral and crystalline order parameters respectively. Bulk mean square displacement is scaled to be at 1. Perfect structures are on the edge of each plot. The lines are a guide for the eye, stressing the collapse of the $w_6$-mobility at all volume fractions in {\bf b} and the absence of collapse in {\bf c}. The scattering at low volume fractions is due to poor averaging of rare structures. Example of \textbf{d} crystal-like cluster and \textbf{e} distorted icosahedron at the respective threshold values.}
	\label{fig:msd_Q6_w6}
\end{figure}
\tikzset{external/force remake=false}

\begin{figure}
	\centering
	\begin{tikzpicture}[%
		pic3d/.style={inner sep=0}, %
		lab/.style={below left=0.5em and 0.5em, text height=0.8em, text depth=0.2em, font=\Large\bfseries},%
		arr/.style={<->, thick, yellow!75!black}%
		]%
	\definecolor{turquoise}{rgb}{0.678431,0.917647,0.917647}
	\definecolor{gold}{rgb}{0.8, 0.498039, 0.196078}
	\node[right=0.01\textwidth, pic3d] (all) {\includegraphics[width=0.48\textwidth]{mrco_ico_scale_go1-0023.png}};
	\node[left=0.01\textwidth, pic3d] at (all.west) (X) {\includegraphics[width=0.48\textwidth]{X_go1.png}};
	\node[below=0.01\textwidth, pic3d] at (all.south) (dyn) {\includegraphics[width=0.48\textwidth]{cgsd_2tau.png}};
	\node[left=0.01\textwidth, pic3d] at (dyn.west) (mrco) {\includegraphics[width=0.48\textwidth]{mrco24_scale_go1_t040_t048.png}};
	\draw [help lines, step=0.12\textwidth, shift=(mrco.south west)] (0, 0) grid (0.48\textwidth, 0.48\textwidth);
	\draw [help lines, step=0.12\textwidth, shift=(dyn.south west)] (0, 0) grid (0.48\textwidth, 0.48\textwidth);
	\matrix[%
		above left=0.01\textwidth of X.south east, fill=gray!25!white, semitransparent, font=\footnotesize,% 
		column 1/.style={right, text height=0.8em, text depth=0.2em},%
		column 2/.style={left, circle, shade, inner sep=0.008\textwidth},%
		nodes={fill opacity=1}
		] (l)
	{
		\node{Crystal}; & \node[ball color=gold!75!black] {};\\
		\node{Icosahedra}; & \node[ball color=red!50!blue] (b2) {};\node[ball color=red!75!black, left] at (b2.west) {}; \node[ball color=blue!75!black, right] at (b2.east) {};\\
		\node{Crystal-like}; & \node[ball color=green!66!black] {}; \\
		\node{Slow}; & \node[ball color=turquoise!75!black] {}; \\
	};
	\node [lab] at (X.north east) {a};
	\node [lab] at (all.north east) {b};
	\node [lab] at (mrco.north east) {c};
	\node [lab] at (dyn.north east) {d};
	\end{tikzpicture}
	\caption{\textbf{Computer reconstruction from confocal microscopy coordinates in our most deeply supercooled sample.} Depth is $\sim 12\sigma$. Only particles of interest and their neighbours are displayed. Each particle is plotted with its real radius. \textbf{a} Particles with more than $7$ crystalline bonds. \textbf{b,} A typical configuration of bond ordered particles. Two icosahedral particle are shown in the same shade if they belong to the same cluster. If a particle is neighbouring both kind of structures, it is displayed as icosahedral. \textbf{c,} Crystal-like particles alone (the order parameter was averaged over $t^{dh}/2$). \textbf{d,} Slow particles (see text). Due to particles going in and out of the field of view, the edges of \textbf{c} and \textbf{d} were not accurate and have been removed.}
	\label{fig:3D}
\end{figure}
\tikzset{external/force remake=false}
\begin{figure}
	\centering
	\begin{tikzpicture}[%
		pic3d/.style={inner sep=0}, %
		lab/.style={below left=0.5em and 0.5em, text height=0.8em, text depth=0.2em, font=\Large\bfseries},%
		]%
	\node[above=0.01\textwidth, pic3d] (X) {\includegraphics[width=0.48\textwidth]{X_go1.png}};
	\node[below=0.01\textwidth, pic3d] (mrco) {\includegraphics[width=0.48\textwidth]{X_mrco_go1.png}};
	\matrix[%
		above left=0.1\textwidth of X.south east, matrix of nodes, ampersand replacement=\&,%
		draw,%
		%every even column/.style={text height=0.8em, text depth=0.2em},%
		every odd column/.style={circle, shade, inner sep=0.25em, above},%
		]{%
		|[ball color=magenta]| \& Crystal \\
		%|[ball color=cyan]| \& Crystal neighbour \&
		|[ball color=green!66!black]| \& Crystal-like \\
	};
	\node [lab] at (X.north east) {a};
	\node [lab] at (mrco.north east) {b};
	\end{tikzpicture}
	\caption{\textbf{Particles with more than $7$ crystalline bonds.} Particles with $Q_6>Q_6^*$ and their neighbours are also plotted in \textbf{b} for comparison. Crystalline particles are always included in high $Q_6$ regions. However, the fluctuations of the crystal-like order cannot be reduced to sub-critical nucleation.}
	\label{fig:X_3D}
\end{figure}
%\tikzset{external/force remake}
\begin{figure}
	\centering
	\begin{tikzpicture}[
		pic3d/.style={inner sep=0}, %
		lab/.style={above right, text height=0.8em, text depth=0.2em, font=\Large\bfseries}%
		]%
		\node[pic3d] (msli) {\includegraphics[width=0.28\textwidth]{comp2D3D_slices}};
		\node[lab] at (msli.south west) {a};
		\node[pic3d, right] at (msli.east) (m3d) {\includegraphics[width=0.28\textwidth]{comp2D3D_crop}};
		\node[lab] at (m3d.south west) {b};
		\node [pic3d, below right] at (m3d.north east) (cg20) {\includegraphics[width=0.14\textwidth]{comp3D_monoscale_r20_crop}};
		\node[lab] at (cg20.south west) {c};
		\node [pic3d, right] at (cg20.east) (cg25) {\includegraphics[width=0.14\textwidth]{comp3D_monoscale_r25_crop}};
		\node[lab] at (cg25.south west) {d};
		\node [pic3d, right] at (cg25.east) (cg30) {\includegraphics[width=0.14\textwidth]{comp3D_monoscale_r30_crop}};
		\node[lab] at (cg30.south west) {e};
		\node [pic3d, above right] at (m3d.south east) (cg35) {\includegraphics[width=0.14\textwidth]{comp3D_monoscale_r35_crop}};
		\node[lab] at (cg35.south west) {f};
		\node [pic3d, right] at (cg35.east) (cg40) {\includegraphics[width=0.14\textwidth]{comp3D_monoscale_r40_crop}};
		\node[lab] at (cg40.south west) {g};
		\node [pic3d, right] at (cg40.east) (cg45) {\includegraphics[width=0.14\textwidth]{comp3D_monoscale_r45_crop}};
		\node[lab] at (cg45.south west) {h};
	\end{tikzpicture}
	\caption{\textbf{Visualisation of the results of various tracking methods for the same portion of image.} The circles on each picture are identical and result from 2D multiscale tracking of each XY slice of the 3D pictures. \textbf{a} XZ and YZ slices of the image (fake colors). \textbf{b} Multiscale 3D tracking. Spheres are drawn with the radius determined by the tracking methods. \textbf{c-h} Crocker and Grier in 3D with blurring radius increasing from \unit{2}{px} to \unit{4.5}{px} by steps of \unit{0.5}{px} (Sphere radii correspond to the blurring radius).}
	\label{fig:localise}
\end{figure}
%\tikzset{external/force remake}
\begin{figure}
	\centering
	\begin{tikzpicture}[%
		pic3d/.style={inner sep=0}, %
		lab/.style={below left, text height=0.8em, text depth=0.2em, font=\Large\bfseries}%
		]
	\begin{axis}[%
		name=hist,
		width=0.45\textwidth,%
		%scale only axis,
		xmin=1, xmax=5,
		axis y line*=left,
		ymin=0, ytick=\empty,%
		ylabel={Size distribution (a.u.)},%
		ylabel near ticks,
		]
		\addplot[ybar, ybar interval] file {SEM_size_distrib.txt};
	\end{axis}
	\begin{axis}[%
		%scale only axis,
		width=0.45\textwidth,%
		xmin=1, xmax=5,
		axis y line*=right,
		xlabel={Diameters [$\micro\metre$]},%
		ymin=0, ytick=\empty,%
		no marks,%
		]
		\addplot+[dashed] file {go1_intensity.sizes};
		\addplot table [x expr ={\thisrowno{0}/1.23}, y index=1] {go1_intensity.sizes};
		\draw[->, ultra thick] (axis cs:3.15,0.05) -- (axis cs: 3.9, 0.05) node[right] {swelling};
		\node[lab] at (rel axis cs:0.95,0.95) {c};
	\end{axis}
	\node[pic3d, below left=0 of hist.left of north west] (XY) {\includegraphics[width=0.14\textwidth]{sliceXY}};
	\node[lab, below=-0.2em of XY.south east, anchor=north east] {a};
	\node[pic3d, below=1.4em of XY] (XZ) {\includegraphics[width=0.14\textwidth]{sliceXZ}};
	\node[lab, below=0 of XZ.south east, anchor=north east] {b};
	\draw[black, ultra thick] (XY.south west) ++(0,-0.2em) -- ++(0.036951014\textwidth,0) node[pos=0,below right=0.2em and 0,inner sep=0] {\tiny\unit{10}{\micro\metre}};
	
	\begin{axis}[%
		at={(hist.right of south east)},
		anchor=left of south west,%
		width=0.45\textwidth,%
		xmin=0, xmax=1.5,%
		xlabel ={$r/\left\langle \sigma\right\rangle$, $\hat{r}$},%
		ymin=0,%
		ylabel=$g(r)$, ylabel near ticks,%
		no marks,%
		legend style={legend pos=north west}
		]
		\addplot+[dashed] table [x expr ={\thisrowno{0}/13}, y index=1] {go1.rdf};
		\addplot table [x expr ={\thisrowno{0}/0.9096}, y expr={\thisrowno{1}*7.5}] {go1.srdf};
		\legend{$g(r)$, $g(\hat{r})$};
		\node[lab] at (rel axis cs:0.95,0.95) {d};
	\end{axis}
	\end{tikzpicture}
	\caption{\textbf{Sizing of our colloids.} \textbf{a,b} XY and XZ slices (detail) of a typical confocal 3D image of our sample. Note the excellent Z resolution, almost not affected by the point spread function. \textbf{c,} Size distribution estimated \emph{in situ} (dashed line) by our multiscale algorithm ($\sim 1.7\times 10^6$ instantaneous sizing). Comparison with the size distribution estimated from \textsc{sem} of only $140$ dry particles (steps) is possible once $23\%$ of swelling of particle diameters is taken into account (full line). \textbf{d,} First peak of the radial distribution function with (full line) and without (dashed) the individual sizes data. Taking into account the measured sizes rectifies the effect of the polydispersity: the peak is thinner and higher.}
	\label{fig:sizing}
\end{figure}
\tikzset{external/force remake=false}
\begin{figure}
	\centering
	\begin{tikzpicture}[lab/.style={below, text height=0.8em, text depth=0.2em, font=\Large\bfseries}]
	\pgfplotsset{ every axis/.append style={ width=0.45\textwidth, cycle list name=black white }}
	\pgfplotsset{cycle list name=black white}
	\begin{loglogaxis}[%
		xmin=10, xmax=1000, xlabel={Cluster size},%
		ymin=1, ylabel={Number of clusters},%
		]
		\addplot+[only marks] table {go1_w6_0012_sizes_rgs_noROI.hist};
		\addplot+[no marks, domain=50:500] {1e6*x^(-2.1)};
		\node[above right] at (axis cs:100,100) {$x^{-2.1}$};
		\node[below left] at (rel axis cs:0.95,0.8) {$w_6<-0.012$};
		\node[lab] at (rel axis cs:0.5, 0.95) {a};
	\end{loglogaxis}
	\begin{loglogaxis}[%
		at={(0.5\textwidth,0)},%
		xmin=10, xmax=1000, xlabel={Cluster size},%
		ymin=1, ylabel={$R_g/\sigma $},%
		ytick={1,10}, minor y tick num=10,%
		yticklabel={%
			\pgfmathfloatparsenumber{\tick}%
			\pgfmathfloatexp{\pgfmathresult}%
			\pgfmathprintnumber{\pgfmathresult}%
		},%
		]
		\addplot+[only marks] table[y expr=\thisrowno{2}/6.14] {go1_w6_0012_sizes_rgs_noROI.hist};
		\addplot+[no marks, domain=25:100] {0.75*x^(0.5)};
		\node[left] at (axis cs:40,6) {$d_f=2$};
		\addplot+[no marks, domain=100:1000] {x^(1/3.)};
		\node[below right] at (axis cs:100,5) {$d_f=3$};
		\addplot+[no marks, domain=20:30] {0.1*x};
		\node[below right] at (axis cs:30,3) {$d_f=1$};
		\node[below right] at (rel axis cs:0.05,0.8) {$w_6<-0.012$};
		\node[lab] at (rel axis cs:0.5, 0.95) {b};
	\end{loglogaxis}
	\begin{axis}[%
		at={(0, -0.36\textwidth)},%
		xlabel=$10^2 \cdot w_6$,%
		xmin=-0.03, xmax=0, xtick={-0.03, -0.02, -0.01}, xticklabels={$-3$, $-2$, $-1$}, minor x tick num=1,%
		xtick scale label code/.code={},%
		extra tick style={grid=major},%
		extra x ticks={\wstar, -0.00782},%
		extra x tick labels={,},%
		ymin=0,%
		ylabel={Largest cluster size},%
		]
		\addplot+[only marks] table {go1_w6.perco};
		\node [left] at (axis cs:\wstar,1e3) {\footnotesize $w_6^*$};
		\node [right] at (axis cs:-0.00782,1e3) {\footnotesize $w_6^{dod}$};
		\node[lab] at (rel axis cs:0.5, 0.95) {c};
	\end{axis}
	\begin{axis}[%
		at={(0.5\textwidth, -0.36\textwidth)},%
		xlabel={Fraction of activated particles},%
		xmin=0, xmax=0.6,ymin=0,%
		ylabel={Largest cluster size},%
		axis y line*=left,%
		]
		\addplot+[only marks] table [x index=2] {go1_w6.perco};
	\end{axis}
	\begin{axis}[%
		at={(0.5\textwidth, -0.36\textwidth)},%
		%xlabel={Fraction of activated particles},%
		xmin=0, xmax=0.6, ymin=0,%
		axis x line=none,%
		axis y line*=right, ytick=\empty,%
		ylabel={Percolation probability (a.u.)},%
		ylabel near ticks,%
		]
		\addplot+[only marks] coordinates{(1, 0)};
		\addplot[const plot, gray, fill=gray, semitransparent, area legend] file {go1_percol_thrs.hist};
		\legend{{by $w_6$},random}
		\node[lab] at (rel axis cs:0.5, 0.95) {d};
	\end{axis}
	\end{tikzpicture}
	\caption{\textbf{Imperfect icosahedral network.} \textbf{a} Cluster size distribution (compatible with random percolation) and \textbf{b} radius of gyration (indicating a fractal dimension $d_f\approx 2$) for nearly percolating threshold ($w_6<-0.012$). \textbf{c} Size of the largest non percolating cluster function of the threshold on $w_6$. Percolation takes place near $w_6\approx -0.011$. \textbf{d} translates \textbf{c} in terms of fraction of activated particles (dots) and also shows the probability of the onset of percolation when particles are randomly activated (line). Activating particles function of icosahedral order or randomly do not change the position of the percolation ($\approx7.5\%$ of the particles activated).}
	\label{fig:percolation}
\end{figure}

\tikzset{external/force remake=false}
\begin{figure}
	\centering
	\begin{tikzpicture}[lab/.style={below right, text height=0.8em, text depth=0.2em, font=\Large\bfseries}]
	\pgfplotsset{fitc/.style={no markers, forget plot, domain=0.3:3}}
	\begin{loglogaxis}[%
		xlabel={$q\sigma$}, ylabel={$S_4(q)$},
		cycle list name=mark list,%
		xmin=0.2, xmax=10, ymin=0.06,%
		width=0.45\textwidth,%
		legend pos=south west, reverse legend, legend style={font=\scriptsize}, %
		]
		\addplot+[blue, fitc] {0.634/(1+0.703^2*x^2)};
		\addplot+[blue, only marks] file {LS3954.S4};
		\addplot+[blue!30!gray, fitc] {0.744/(1+0.804^2*x^2)};
		\addplot+[blue!30!gray, only marks] file {LS4446.S4};
		\addplot+[gray, fitc] {0.778/(1+0.835^2*x^2)};
		\addplot+[gray, only marks] file {LS4582.S4};
		\addplot+[red!50!gray, fitc] {1.011/(1+1.017^2*x^2)};
		\addplot+[red!50!gray, only marks] file {LS5079.S4};
		\addplot+[red, fitc] {2.950/(1+1.959^2*x^2)};
		\addplot+[red, only marks] file {go1.S4};
		\addlegendimage{legend image code/.code={\node[right] {$\phi\;\pm$};}};
		\legend{$0.497$, $0.535$, $0.540$, $0.555$, $0.575$, $0.003$};
		\node[lab] at (rel axis cs:0, 1) {a};
	\end{loglogaxis}
	\begin{loglogaxis}[%
		at={(0.5\textwidth,0)},
		width=0.45\textwidth,%
		xlabel={$q\xi_4$}, ylabel={$S_4(q)/\chi_4$},
		cycle list name=mark list,%
		xmin=0.25, xmax=5, ymin=0.05,%
		xtick={0.5, 1, 2, 4},
		xticklabel=\pgfmathparse{exp(\tick)}\pgfmathprintnumber{\pgfmathresult},
		legend pos=south west, reverse legend, legend style={font=\scriptsize}, %
		]
		\addplot+[blue, only marks] table[x expr={\thisrowno{0}*0.703}, y expr={\thisrowno{1}/0.634}] {LS3954.S4};
		\addplot+[blue!30!gray, only marks] table[x expr={\thisrowno{0}*0.804}, y expr={\thisrowno{1}/0.744}] {LS4446.S4};
		\addplot+[gray, only marks] table[x expr={\thisrowno{0}*0.835}, y expr={\thisrowno{1}/0.778}] {LS4582.S4};
		\addplot+[red!50!gray, only marks] table[x expr={\thisrowno{0}*1.017}, y expr={\thisrowno{1}/1.011}] {LS5079.S4};
		\addplot+[red, only marks] table[x expr={\thisrowno{0}*1.959}, y expr={\thisrowno{1}/2.950}] {go1.S4};
		\addplot+[black, fitc] {1/(1+x^2)};
		\addlegendimage{legend image code/.code={\node[right] {$\phi\;\pm$};}};
		\legend{$0.497$, $0.535$, $0.540$, $0.555$, $0.575$, $0.003$};
		\node[lab] at (rel axis cs:0, 1) {b};
	\end{loglogaxis}
	\end{tikzpicture}
	\caption{\textbf{Four-point structure factor} at different volume fractions}
	\label{fig:S4}
\end{figure}

\end{document}